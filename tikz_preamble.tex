\usepackage{tikz}
\usetikzlibrary{fit}					% fitting shapes to coordinates
\usetikzlibrary{backgrounds}	% drawing the background after the foreground
\usetikzlibrary{calc, arrows, fit, positioning, patterns, decorations.pathreplacing, shapes}
\usetikzlibrary{shadows}
\usepackage{xifthen}
\usetikzlibrary{arrows.meta}
\usepackage{smartdiagram} % used in the applications slide

\tikzstyle{var}=[circle,
                 thick,
                 draw=black!100,
                 font=\small,
                ]

\tikzstyle{bag}=[circle,
                 ultra thick,
                 draw=black!100,
                 font=\small,
                ]

\tikzstyle{hsepset}=[rectangle,
                     minimum width=1.3cm,
                     draw=black!100,
                     font=\small,
                    ]

\tikzstyle{vsepset}=[rectangle,
                     minimum width=1.3cm,
                     draw=black!100,
                     rotate=90,
                     font=\small,
                    ]

\tikzstyle{msgcircle}=[shape=circle,
                       draw,
                       inner sep=0pt,
                       minimum size=4mm,
                       font=\scriptsize,
                      ]

\tikzstyle{roundrect}=[shape=rectangle,
                       draw,
                       rounded corners,
                       inner sep=7pt,
                       minimum width=22mm,
                       minimum height=9mm,
                       font=\footnotesize,
                       %thick,
                      ]

% Used in the factor graph
\tikzstyle{box} = [draw, minimum size=7.0mm]
\tikzstyle{line} = [draw, -latex,>=latex]

\newcommand{\msgcircle}[7]{
  % Circle left arrow down
  \ifthenelse{\isin{#1}{left} \AND \isin{#2}{down}}{
    \coordinate (anchor) at ($({#3})!{#5}!({#4})$);
    \node[color={#7}, msgcircle, xshift=-5.0mm] at (anchor) {#6};
    \node[color={#7}, xshift=-1.5mm] at (anchor) {$\Big\downarrow$};
  }{}
  % Circle right arrow down
  \ifthenelse{\isin{#1}{right} \AND \isin{#2}{down}}{
    \coordinate (anchor) at ($({#3})!{#5}!({#4})$);
    \node[color={#7}, msgcircle ,xshift=5.0mm] at (anchor) {#6};
    \node[color={#7}, xshift=1.7mm] at (anchor) {$\Big\downarrow$};
  }{}

  % Circle down arrow right
  \ifthenelse{\isin{#1}{down} \AND \isin{#2}{right}}{
    \coordinate (anchor) at ($({#3})!{#5}!({#4})$);
    \node[color={#7}, msgcircle, yshift=-5.0mm] at (anchor) {#6};
    \node[color={#7}, yshift=-1.7mm, rotate=90] at (anchor) {$\Big\downarrow$};
  }{}
  % Circle up arrow right
  \ifthenelse{\isin{#1}{up} \AND \isin{#2}{right}}{
    \coordinate (anchor) at ($({#3})!{#5}!({#4})$);
    \node[color={#7}, msgcircle, yshift=5.0mm] at (anchor) {#6};
    \node[color={#7}, yshift=1.5mm, rotate=90] at (anchor) {$\Big\downarrow$};
  }{}

  % Circle down arrow left
  \ifthenelse{\isin{#1}{down} \AND \isin{#2}{left}}{
    \coordinate (anchor) at ($({#3})!{#5}!({#4})$);
    \node[color={#7}, msgcircle, yshift=-5.0mm] at (anchor) {#6};
    \node[color={#7}, yshift=-1.7mm, rotate=-90] at (anchor) {$\Big\downarrow$};
  }{}
  % Circle up arrow left
  \ifthenelse{\isin{#1}{up} \AND \isin{#2}{left}}{
    \coordinate (anchor) at ($({#3})!{#5}!({#4})$);
    \node[color={#7}, msgcircle, yshift=5.0mm] at (anchor) {#6};
    \node[color={#7}, yshift=1.5mm, rotate=-90] at (anchor) {$\Big\downarrow$};
  }{}

  % Circle left arrow up
  \ifthenelse{\isin{#1}{left} \AND \isin{#2}{up}}{
    \coordinate (anchor) at ($({#3})!{#5}!({#4})$);
    \node[color={#7}, msgcircle, xshift=-5.0mm] at (anchor) {#6};
    \node[color={#7}, xshift=-1.5mm, rotate=180] at (anchor) {$\Big\downarrow$};
  }{}
  % Circle right arrow up
  \ifthenelse{\isin{#1}{right} \AND \isin{#2}{up}}{
    \coordinate (anchor) at ($({#3})!{#5}!({#4})$);
    \node[color={#7}, msgcircle, xshift=5.0mm] at (anchor) {#6};
    \node[color={#7}, xshift=1.7mm, rotate=180] at (anchor) {$\Big\downarrow$};
  }{}
}

% https://tex.stackexchange.com/questions/39553/how-can-i-show-coordinates-by-grid-in-tikz-automatically
\makeatletter
\def\grd@save@target#1{%
  \def\grd@target{#1}}
\def\grd@save@start#1{%
  \def\grd@start{#1}}
\tikzset{
  grid with coordinates/.style={
    to path={%
      \pgfextra{%
        \edef\grd@@target{(\tikztotarget)}%
        \tikz@scan@one@point\grd@save@target\grd@@target\relax
        \edef\grd@@start{(\tikztostart)}%
        \tikz@scan@one@point\grd@save@start\grd@@start\relax
        \draw[minor help lines] (\tikztostart) grid (\tikztotarget);
        \draw[major help lines] (\tikztostart) grid (\tikztotarget);
        \grd@start
        \pgfmathsetmacro{\grd@xa}{\the\pgf@x/1cm}
        \pgfmathsetmacro{\grd@ya}{\the\pgf@y/1cm}
        \grd@target
        \pgfmathsetmacro{\grd@xb}{\the\pgf@x/1cm}
        \pgfmathsetmacro{\grd@yb}{\the\pgf@y/1cm}
        \pgfmathsetmacro{\grd@xc}{\grd@xa + \pgfkeysvalueof{/tikz/grid with coordinates/major step}}
        \pgfmathsetmacro{\grd@yc}{\grd@ya + \pgfkeysvalueof{/tikz/grid with coordinates/major step}}
        \foreach \x in {\grd@xa,\grd@xc,...,\grd@xb}
        \node[anchor=north] at (\x,\grd@ya) {\tiny{\pgfmathprintnumber{\x}}};
        \foreach \y in {\grd@ya,\grd@yc,...,\grd@yb}
        \node[anchor=east] at (\grd@xa,\y) {\tiny{\pgfmathprintnumber{\y}}};
      }
    }
  },
  minor help lines/.style={
    help lines,
    step=\pgfkeysvalueof{/tikz/grid with coordinates/minor step}
  },
  major help lines/.style={
    help lines,
    line width=\pgfkeysvalueof{/tikz/grid with coordinates/major line width},
    step=\pgfkeysvalueof{/tikz/grid with coordinates/major step}
  },
  grid with coordinates/.cd,
  minor step/.initial=0,
  major step/.initial=1,
  major line width/.initial=0.5pt,
}
\makeatother


% https://tex.stackexchange.com/questions/82413/drawing-a-node-surrounding-arbitrarily-places-nodes
% https://tex.stackexchange.com/questions/27171/padded-boundary-of-convex-hull#27185
\usetikzlibrary{calc,trees}

\newcommand{\convexpath}[2]{
  [   
  create hullnodes/.code={
    \global\edef\namelist{#1}
    \foreach [count=\counter] \nodename in \namelist {
      \global\edef\numberofnodes{\counter}
      \node at (\nodename) [draw=none,name=hullnode\counter] {};
    }
    \node at (hullnode\numberofnodes) [name=hullnode0,draw=none] {};
    \pgfmathtruncatemacro\lastnumber{\numberofnodes+1}
    \node at (hullnode1) [name=hullnode\lastnumber,draw=none] {};
  },
  create hullnodes
  ]
  ($(hullnode1)!#2!-90:(hullnode0)$)
  \foreach [
  evaluate=\currentnode as \previousnode using \currentnode-1,
  evaluate=\currentnode as \nextnode using \currentnode+1
  ] \currentnode in {1,...,\numberofnodes} {
    -- ($(hullnode\currentnode)!#2!-90:(hullnode\previousnode)$)
    let \p1 = ($(hullnode\currentnode)!#2!-90:(hullnode\previousnode) - (hullnode\currentnode)$),
    \n1 = {atan2(\y1,\x1)},
    \p2 = ($(hullnode\currentnode)!#2!90:(hullnode\nextnode) - (hullnode\currentnode)$),
    \n2 = {atan2(\y2,\x2)},
    \n{delta} = {-Mod(\n1-\n2,360)}
    in 
    {arc [start angle=\n1, delta angle=\n{delta}, radius=#2]}
  }
  -- cycle
}

% A command to create an overview of the junction tree algorithm
% Takes 2 parameters:
% 1. Name of the section to highlight
% 2. Name of the subsection to highlight
\newcommand{\pptc}[2]{

  \tikzset {
    every node/.style={node distance=1.4cm},
    mybox/.style= {
      rectangle,rounded corners,drop shadow,minimum height=0.7cm,font=\small,
      minimum width=\columnwidth*0.8,align=center,fill=white,draw=gray,line width=1pt
    },
    myarrow/.style= { draw=gray,line width=2pt,-stealth },
    mylabel/.style={text=black,right,xshift=2mm,text width=4.2cm,font=\small},
    surrbox/.style={thick,draw=black,rounded corners=2mm},
    surrboxlabel/.style={anchor=center,rotate=90,yshift=3mm,font=\small},
  }

  \begin{tikzpicture}

    \ifthenelse{\equal{#1}{Probabilistic Graphical Model}}{
      \node[mybox] (pgm) {\textbf{\alert{Probabilistic Graphical Model}}};
    }{
      \node[mybox] (pgm) {Probabilistic Graphical Model};
    }

    \ifthenelse{\equal{#1}{Junction Tree} }{
      \node[mybox,below=of pgm,yshift=-0.4cm] (jts) {\textbf{\alert{Junction Tree}}};
    }{
      \node[mybox,below=of pgm,yshift=-0.4cm] (jts) {Junction Tree};
    }

    \ifthenelse{\equal{#1}{Inconsistent Junction Tree}}{
      \node[mybox,below=of jts] (ijt) {\textbf{\alert{Inconsistent Junction Tree}}};
    }{
      \node[mybox,below=of jts] (ijt) {Inconsistent Junction Tree};
    }

    \ifthenelse{\equal{#1}{Consistent Junction Tree}}{
      \node[mybox,below=of ijt,yshift=0.4cm] (cjt) {\textbf{\alert{Consistent Junction Tree}}};
    }{
      \node[mybox,below=of ijt,yshift=0.4cm] (cjt) {Consistent Junction Tree};
    }

    \ifthenelse{\equal{#1}{Consistent Junction Tree}}{
      \node[mybox,below=of ijt,yshift=0.4cm] (cjt) {\textbf{\alert{Consistent Junction Tree}}};
    }{
      \node[mybox,below=of ijt,yshift=0.4cm] (cjt) {Consistent Junction Tree};
    }

    \node[below=of cjt] (mar) {$P(V \mid \bm{E=e})$};

    \draw[myarrow] (pgm) -- (jts);
    \draw[myarrow] (jts) -- (ijt);
    \draw[myarrow] (ijt) -- (cjt);
    \draw[myarrow] (cjt) -- (mar);

    %-------------------------------------------------------------------

    \ifthenelse{\equal{#2}{Moralization}}{
      \path (pgm) -- node[mylabel] (gt) {\textbf{\alert{1. Moralization}}\\2. Triangulation\\3. Connection of clusters} (jts);
    }{}

    \ifthenelse{\equal{#2}{Triangulation}}{
      \path (pgm) -- node[mylabel] (gt) {1. Moralization\\\textbf{\alert{2. Triangulation}}\\3. Connection of clusters} (jts);
    }{}

    \ifthenelse{\equal{#2}{Connection of clusters}}{
      \path (pgm) -- node[mylabel] (gt) {1. Moralization\\2. Triangulation\\\textbf{\alert{3. Connection of clusters}}} (jts);
    }{}

    \ifthenelse{\not \equal{#2}{Moralization} \AND \not \equal{#2}{Triangulation} \AND \not \equal{#2}{Connection of clusters}}{
      \path (pgm) -- node[mylabel] (gt) {1. Moralization\\2. Triangulation\\3. Connection of clusters} (jts);
    }{}

    %-------------------------------------------------------------------

    \ifthenelse{\equal{#2}{Initialization}}{
      \path (jts) -- node[mylabel] (ini) {\textbf{\alert{1. Initialization}}\\2. Observation entry} (ijt);
    }{}

    \ifthenelse{\equal{#2}{Observation entry}}{
      \path (jts) -- node[mylabel] (ini) {1. Initialization\\\textbf{\alert{2. Observation entry}}} (ijt);
    }{}

    \ifthenelse{\not \equal{#2}{Initialization} \AND \not \equal{#2}{Observation entry}}{
      \path (jts) -- node[mylabel] (ini) {1. Initialization\\2. Observation entry} (ijt);
    }{}

    %-------------------------------------------------------------------

    \ifthenelse{\equal{#2}{Propagation}}{
      \path (ijt) -- node[mylabel] (pro) {\textbf{\alert{Propagation}}} (cjt);
    }{
      \path (ijt) -- node[mylabel] (pro) {Propagation} (cjt);
    }

    %-------------------------------------------------------------------

    \ifthenelse{\equal{#2}{Marginalization}}{
      \path (cjt) -- node[mylabel] (mar-nor) {\textbf{\alert{1. Marginalization}}\\2. Normalization} (mar);
    }{}

    \ifthenelse{\equal{#2}{Normalization}}{
      \path (cjt) -- node[mylabel] (mar-nor) {1. Marginalization\\\textbf{\alert{2. Normalization}}} (mar);
    }{}

    \ifthenelse{\not \equal{#2}{Marginalization} \AND \not \equal{#2}{Normalization}}{
      \path (cjt) -- node[mylabel] (mar-nor) {1. Marginalization\\2. Normalization} (mar);
    }{}

  \end{tikzpicture}
}


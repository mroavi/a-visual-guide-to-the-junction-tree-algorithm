\tikzset {
  every node/.style={node distance=20mm and 1.0mm},
  myroundbox/.style= {
    rectangle,rounded corners=3mm,drop shadow,minimum height=1.0cm,font=\small,
    minimum width=\columnwidth*0.28,align=center,fill=white,draw=black,
  },
  myrectbox/.style= {
    rectangle,drop shadow,minimum height=1.0cm,font=\small,
    minimum width=\columnwidth*0.28,align=center,fill=white,draw=black,
  },
  myarrow/.style={=black,-{Stealth[scale=1.0]},shorten >=2pt},
  mylabel/.style={text=black,right,xshift=2mm,text width=3.2cm,font=\small},
  surrbox/.style={thick,draw=black,rounded corners=2mm},
  surrboxlabel/.style={anchor=center,rotate=90,yshift=3mm,font=\small},
}

\begin{tikzpicture}

  %\draw[help lines] (0,0) grid (10,-7);

  % mrv: the "node distances" refer to the distance between the edge of a shape
  % to the edge of the other shape. That is why I use "ie_aux" and "mar_aux"
  % below: to have equal distances between nodes with respect to the center of
  % the shapes.

  % row 1
  \node[myroundbox] (rv) {Random Variables\\$\mathcal{V}$};
  \node[right=of rv](aux1) {};
  \node[right=of aux1,myroundbox] (jd) {Joint Distribution\\$P(\mathcal{V})$};
  \node[right=of jd](aux2) {};
  \node[right=of aux2,myroundbox] (e) {Evidence\\$\bm{E=e}$};
  \node[right=of e](aux3) {};
  \node[right=of aux3,myroundbox] (qv) {Query Variables\\$\bm{Q}$};
  % row 2
  \node[below=of aux2,myrectbox] (ie) {Inference Engine};
  \node[below=of aux2] (ie_aux) {};
  % row 3
  \node[below=of ie_aux,myroundbox] (mar) {$P(\bm{Q} \mid {\bf E=e})$};
  \node[below=of ie_aux] (mar_aux) {};
  % row 0
  \node[above=of aux2,yshift=-12mm] (in) {\textbf{Input}};
  % row 4
  \node[below=of mar_aux,yshift=7mm] (out) {\textbf{Output}};

  %% edges
  \draw[myarrow] (rv) -- (ie);
  \draw[myarrow] (jd) -- (ie);
  \draw[myarrow] (e)  -- (ie);
  \draw[myarrow] (qv) -- (ie);
  \draw[myarrow] (ie) -- (mar);

\end{tikzpicture}


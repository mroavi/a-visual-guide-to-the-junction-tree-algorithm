%------------------------------------------------------------------------------
%%
%------------------------------------------------------------------------------

\begin{frame}[fragile]{Connection of Clusters}

\centering

{\footnotesize
  Let's transform the complete cluster graph into a \textit{junction tree} in an
  \textit{optimal} way \cite{jensen2013optimal}.
}

\begin{figure}

\begin{subfigure}[b]{0.49\linewidth}
\centering
\scalebox{0.8}{
\begin{tikzpicture}[>=latex]
  
  % The various elements are conveniently placed using a matrix:
  \matrix[row sep=1.0cm,column sep=1.0cm,ampersand replacement=\&] {

    % First line
    \node (bd)  [bag, minimum size=1.8cm, draw=D] {$\{B,D\}$};    \&
    \node (bef) [bag, minimum size=1.8cm, draw=F] {$\{B,E,F\}$}; \\
    % Second line
    \node (bce) [bag, minimum size=1.8cm, draw=E] {$\{B,C,E\}$};  \&
    \node (abc) [bag, minimum size=1.8cm, draw=C] {$\{A,B,C\}$}; \\
  };


  %% The diagram elements are now connected through lines:
  \path[-]
    (bd) edge (bef)
    (bd) edge (bce)
    (bd) edge (abc)
    (bef) edge (bce)
    (bef) edge (abc)
    (bce) edge (abc)
    ;

  %% The diagram elements are now connected through lines:
  %% (See manual pg. 132)
  %\path[-]
  %  (bd)  edge node[above] {$1$} (bef)
  %        edge [ultra thick] node[left] {$1$} (bce)
  %        edge node[above right,pos=0.8] {$1$} (abc)
  %  (bef) edge [ultra thick] node[above left,pos=0.8] {$2$} (bce)
  %        edge node[right] {$1$} (abc)
  %  (bce) edge [ultra thick] node[below] {$2$} (abc)
  %  ;

\end{tikzpicture}
}
\end{subfigure}

\end{figure}

\end{frame}

%------------------------------------------------------------------------------
%%
%------------------------------------------------------------------------------

\begin{frame}[fragile]{Connection of Clusters}

\centering

{\footnotesize
  Like before, subscripts denote the variable's \textit{cardinality}.
}

\begin{figure}

\begin{subfigure}[b]{0.49\linewidth}
\centering
\scalebox{0.8}{
\begin{tikzpicture}[>=latex]
  
  % The various elements are conveniently placed using a matrix:
  \matrix[row sep=1.0cm,column sep=1.0cm,ampersand replacement=\&] {

    % First line
    \node (bd)  [bag, minimum size=1.8cm, draw=D] {$\{B_4,D_7\}$};    \&
    \node (bef) [bag, minimum size=1.8cm, draw=F] {$\{B_4,E_2,F_3\}$}; \\
    % Second line
    \node (bce) [bag, minimum size=1.8cm, draw=E] {$\{B_4,C_3,E_2\}$};  \&
    \node (abc) [bag, minimum size=1.8cm, draw=C] {$\{A_3,B_4,C_3\}$}; \\
  };

  %% The diagram elements are now connected through lines:
  \path[-]
    (bd) edge (bef)
    (bd) edge (bce)
    (bd) edge (abc)
    (bef) edge (bce)
    (bef) edge (abc)
    (bce) edge (abc)
    ;

\end{tikzpicture}
}
\end{subfigure}

\end{figure}

\end{frame}

%------------------------------------------------------------------------------
%%
%------------------------------------------------------------------------------

\begin{frame}[fragile]{Connection of Clusters}

\centering

{\footnotesize
  Count the common variables between each pair of clusters.
}

\begin{figure}

\begin{subfigure}[b]{0.49\linewidth}
\centering
\scalebox{0.8}{
\begin{tikzpicture}[>=latex]
  
  % The various elements are conveniently placed using a matrix:
  \matrix[row sep=1.0cm,column sep=1.0cm,ampersand replacement=\&] {

    % First line
    \node (bd)  [bag, minimum size=1.8cm, draw=D] {$\{B_4,D_7\}$};    \&
    \node (bef) [bag, minimum size=1.8cm, draw=F] {$\{B_4,E_2,F_3\}$}; \\
    % Second line
    \node (bce) [bag, minimum size=1.8cm, draw=E] {$\{B_4,C_3,E_2\}$};  \&
    \node (abc) [bag, minimum size=1.8cm, draw=C] {$\{A_3,B_4,C_3\}$}; \\
  };


  % The diagram elements are now connected through lines:
  % (See manual pg. 132)
  \path[-]
    (bd)  edge node[above] {$1$} (bef)
          edge node[left] {$1$} (bce)
          edge node[above right,pos=0.8] {$1$} (abc)
    (bef) edge node[above left,pos=0.8] {$2$} (bce)
          edge node[right] {$1$} (abc)
    (bce) edge node[below] {$2$} (abc)
    ;

\end{tikzpicture}
}
\end{subfigure}

\end{figure}

\end{frame}

%------------------------------------------------------------------------------
%%
%------------------------------------------------------------------------------

\begin{frame}[fragile]{Connection of Clusters}

\centering

{\footnotesize
  Create a new graph with only the nodes of the complete cluster graph.
}

\begin{figure}

\begin{subfigure}[b]{0.49\linewidth}
\centering
\scalebox{0.8}{
\begin{tikzpicture}[>=latex]
  
  % The various elements are conveniently placed using a matrix:
  \matrix[row sep=1.0cm,column sep=1.0cm,ampersand replacement=\&] {

    % First line
    \node (bd)  [bag, minimum size=1.8cm, draw=D] {$\{B_4,D_7\}$};    \&
    \node (bef) [bag, minimum size=1.8cm, draw=F] {$\{B_4,E_2,F_3\}$}; \\
    % Second line
    \node (bce) [bag, minimum size=1.8cm, draw=E] {$\{B_4,C_3,E_2\}$};  \&
    \node (abc) [bag, minimum size=1.8cm, draw=C] {$\{A_3,B_4,C_3\}$}; \\
  };

  % The diagram elements are now connected through lines:
  % (See manual pg. 132)
  \path[-]
    (bd)  edge node[above] {$1$} (bef)
          edge node[left] {$1$} (bce)
          edge node[above right,pos=0.8] {$1$} (abc)
    (bef) edge node[above left,pos=0.8] {$2$} (bce)
          edge node[right] {$1$} (abc)
    (bce) edge node[below] {$2$} (abc)
    ;

\end{tikzpicture}
}
\end{subfigure}
\hfill
\begin{subfigure}[b]{0.49\linewidth}
\centering
\scalebox{0.8}{
\begin{tikzpicture}[>=latex]
  
  % The various elements are conveniently placed using a matrix:
  \matrix[row sep=1.0cm,column sep=1.0cm,ampersand replacement=\&] {

    % First line
    \node (bd)  [bag, minimum size=1.8cm, draw=D] {$\{B_4,D_7\}$};    \&
    \node (bef) [bag, minimum size=1.8cm, draw=F] {$\{B_4,E_2,F_3\}$}; \\
    % Second line
    \node (bce) [bag, minimum size=1.8cm, draw=E] {$\{B_4,C_3,E_2\}$};  \&
    \node (abc) [bag, minimum size=1.8cm, draw=C] {$\{A_3,B_4,C_3\}$}; \\
  };
\end{tikzpicture}
}
\end{subfigure}

\end{figure}

\end{frame}

%------------------------------------------------------------------------------
%%
%------------------------------------------------------------------------------

\begin{frame}[fragile]{Connection of Clusters}

\centering

{\footnotesize
  Select the edge that connects the two clusters with the most common variables
  \textit{and} that would not create a loop in the right graph if moved.
}

\begin{figure}

\begin{subfigure}[b]{0.49\linewidth}
\centering
\scalebox{0.8}{
\begin{tikzpicture}[>=latex]
  
  % The various elements are conveniently placed using a matrix:
  \matrix[row sep=1.0cm,column sep=1.0cm,ampersand replacement=\&] {

    % First line
    \node (bd)  [bag, minimum size=1.8cm, draw=D] {$\{B_4,D_7\}$};    \&
    \node (bef) [bag, minimum size=1.8cm, draw=F] {$\{B_4,E_2,F_3\}$}; \\
    % Second line
    \node (bce) [bag, minimum size=1.8cm, draw=E] {$\{B_4,C_3,E_2\}$};  \&
    \node (abc) [bag, minimum size=1.8cm, draw=C] {$\{A_3,B_4,C_3\}$}; \\
  };

  % The diagram elements are now connected through lines:
  % (See manual pg. 132)
  \path[-]
    (bd)  edge node[above] {$1$} (bef)
          edge node[left] {$1$} (bce)
          edge node[above right,pos=0.8] {$1$} (abc)
    (bef) edge node[above left,pos=0.8] {$2$} (bce)
          edge node[right] {$1$} (abc)
    (bce) edge node[below] {$2$} (abc)
    ;

\end{tikzpicture}
}
\end{subfigure}
\hfill
\begin{subfigure}[b]{0.49\linewidth}
\centering
\scalebox{0.8}{
\begin{tikzpicture}[>=latex]
  
  % The various elements are conveniently placed using a matrix:
  \matrix[row sep=1.0cm,column sep=1.0cm,ampersand replacement=\&] {

    % First line
    \node (bd)  [bag, minimum size=1.8cm, draw=D] {$\{B_4,D_7\}$};    \&
    \node (bef) [bag, minimum size=1.8cm, draw=F] {$\{B_4,E_2,F_3\}$}; \\
    % Second line
    \node (bce) [bag, minimum size=1.8cm, draw=E] {$\{B_4,C_3,E_2\}$};  \&
    \node (abc) [bag, minimum size=1.8cm, draw=C] {$\{A_3,B_4,C_3\}$}; \\
  };
\end{tikzpicture}
}
\end{subfigure}

\end{figure}

\end{frame}

%------------------------------------------------------------------------------
%%
%------------------------------------------------------------------------------

\begin{frame}[fragile]{Connection of Clusters}

\centering

{\footnotesize
  Tie.\\
  Then select the edge connecting the clusters that have the smallest
  \textit{state space} size.
}

\begin{figure}

\begin{subfigure}[b]{0.49\linewidth}
\centering
\scalebox{0.8}{
\begin{tikzpicture}[>=latex]
  
  % The various elements are conveniently placed using a matrix:
  \matrix[row sep=1.0cm,column sep=1.0cm,ampersand replacement=\&] {

    % First line
    \node (bd)  [bag, minimum size=1.8cm, draw=D] {$\{B_4,D_7\}$};    \&
    \node (bef) [bag, minimum size=1.8cm, draw=F] {$\{B_4,E_2,F_3\}$}; \\
    % Second line
    \node (bce) [bag, minimum size=1.8cm, draw=E] {$\{B_4,C_3,E_2\}$};  \&
    \node (abc) [bag, minimum size=1.8cm, draw=C] {$\{A_3,B_4,C_3\}$}; \\
  };

  % The diagram elements are now connected through lines:
  % (See manual pg. 132)
  \path[-]
    (bd)  edge node[above] {$1$} (bef)
          edge node[left] {$1$} (bce)
          edge node[above right,pos=0.8] {$1$} (abc)
    (bef) edge [ultra thick] node[above left,pos=0.8] {$2$} (bce)
          edge node[right] {$1$} (abc)
    (bce) edge [ultra thick] node[below] {$2$} (abc)
    ;

\end{tikzpicture}
}
\end{subfigure}
\hfill
\begin{subfigure}[b]{0.49\linewidth}
\centering
\scalebox{0.8}{
\begin{tikzpicture}[>=latex]
  
  % The various elements are conveniently placed using a matrix:
  \matrix[row sep=1.0cm,column sep=1.0cm,ampersand replacement=\&] {

    % First line
    \node (bd)  [bag, minimum size=1.8cm, draw=D] {$\{B_4,D_7\}$};    \&
    \node (bef) [bag, minimum size=1.8cm, draw=F] {$\{B_4,E_2,F_3\}$}; \\
    % Second line
    \node (bce) [bag, minimum size=1.8cm, draw=E] {$\{B_4,C_3,E_2\}$};  \&
    \node (abc) [bag, minimum size=1.8cm, draw=C] {$\{A_3,B_4,C_3\}$}; \\
  };
\end{tikzpicture}
}
\end{subfigure}

\end{figure}

\end{frame}

%------------------------------------------------------------------------------
%%
%------------------------------------------------------------------------------

\begin{frame}[fragile]{Connection of Clusters}

\centering

{\footnotesize
  $ \lvert \{B_4,C_3,E_2\} \rvert + \lvert \{B_4,E_2,F_3\} \rvert <
  \lvert \{B_4,C_3,E_2\} \rvert + \lvert \{A_3,B_4,C_3\} \rvert $,\\
  therefore the edge connecting
  $\{B_4,C_3,E_2\}$ and $\{B_4,E_2,F_3\}$ wins.
}

\begin{figure}

\begin{subfigure}[b]{0.49\linewidth}
\centering
\scalebox{0.8}{
\begin{tikzpicture}[>=latex]
  
  % The various elements are conveniently placed using a matrix:
  \matrix[row sep=1.0cm,column sep=1.0cm,ampersand replacement=\&] {

    % First line
    \node (bd)  [bag, minimum size=1.8cm, draw=D] {$\{B_4,D_7\}$};    \&
    \node (bef) [bag, minimum size=1.8cm, draw=F] {$\{B_4,E_2,F_3\}$}; \\
    % Second line
    \node (bce) [bag, minimum size=1.8cm, draw=E] {$\{B_4,C_3,E_2\}$};  \&
    \node (abc) [bag, minimum size=1.8cm, draw=C] {$\{A_3,B_4,C_3\}$}; \\
  };

  % The diagram elements are now connected through lines:
  % (See manual pg. 132)
  \path[-]
    (bd)  edge node[above] {$1$} (bef)
          edge node[left] {$1$} (bce)
          edge node[above right,pos=0.8] {$1$} (abc)
    (bef) edge [ultra thick] node[above left,pos=0.8] {$2$} (bce)
          edge node[right] {$1$} (abc)
    (bce) edge node[below] {$2$} (abc)
    ;

\end{tikzpicture}
}
\end{subfigure}
\hfill
\begin{subfigure}[b]{0.49\linewidth}
\centering
\scalebox{0.8}{
\begin{tikzpicture}[>=latex]
  
  % The various elements are conveniently placed using a matrix:
  \matrix[row sep=1.0cm,column sep=1.0cm,ampersand replacement=\&] {

    % First line
    \node (bd)  [bag, minimum size=1.8cm, draw=D] {$\{B_4,D_7\}$};    \&
    \node (bef) [bag, minimum size=1.8cm, draw=F] {$\{B_4,E_2,F_3\}$}; \\
    % Second line
    \node (bce) [bag, minimum size=1.8cm, draw=E] {$\{B_4,C_3,E_2\}$};  \&
    \node (abc) [bag, minimum size=1.8cm, draw=C] {$\{A_3,B_4,C_3\}$}; \\
  };
\end{tikzpicture}
}
\end{subfigure}

\end{figure}

\end{frame}

%------------------------------------------------------------------------------
%%
%------------------------------------------------------------------------------

\begin{frame}[fragile]{Connection of Clusters}

\centering

{\footnotesize
  Move the selected edge to the right graph.
}

\begin{figure}

\begin{subfigure}[b]{0.49\linewidth}
\centering
\scalebox{0.8}{
\begin{tikzpicture}[>=latex]
  
  % The various elements are conveniently placed using a matrix:
  \matrix[row sep=1.0cm,column sep=1.0cm,ampersand replacement=\&] {

    % First line
    \node (bd)  [bag, minimum size=1.8cm, draw=D] {$\{B_4,D_7\}$};    \&
    \node (bef) [bag, minimum size=1.8cm, draw=F] {$\{B_4,E_2,F_3\}$}; \\
    % Second line
    \node (bce) [bag, minimum size=1.8cm, draw=E] {$\{B_4,C_3,E_2\}$};  \&
    \node (abc) [bag, minimum size=1.8cm, draw=C] {$\{A_3,B_4,C_3\}$}; \\
  };

  % The diagram elements are now connected through lines:
  % (See manual pg. 132)
  \path[-]
    (bd)  edge node[above] {$1$} (bef)
          edge node[left] {$1$} (bce)
          edge node[above right,pos=0.8] {$1$} (abc)
    %(bef) edge node[above left,pos=0.8] {$2$} (bce)
    (bef) edge node[right] {$1$} (abc)
    (bce) edge node[below] {$2$} (abc)
    ;

\end{tikzpicture}
}
\end{subfigure}
\hfill
\begin{subfigure}[b]{0.49\linewidth}
\centering
\scalebox{0.8}{
\begin{tikzpicture}[>=latex]
  
  % The various elements are conveniently placed using a matrix:
  \matrix[row sep=1.0cm,column sep=1.0cm,ampersand replacement=\&] {

    % First line
    \node (bd)  [bag, minimum size=1.8cm, draw=D] {$\{B_4,D_7\}$};    \&
    \node (bef) [bag, minimum size=1.8cm, draw=F] {$\{B_4,E_2,F_3\}$}; \\
    % Second line
    \node (bce) [bag, minimum size=1.8cm, draw=E] {$\{B_4,C_3,E_2\}$};  \&
    \node (abc) [bag, minimum size=1.8cm, draw=C] {$\{A_3,B_4,C_3\}$}; \\
  };
  \path[-]
    (bce) edge (bef)
    ;
\end{tikzpicture}
}
\end{subfigure}

\end{figure}

\end{frame}

%------------------------------------------------------------------------------
%%
%------------------------------------------------------------------------------

\begin{frame}[fragile]{Connection of Clusters}

\centering

{\footnotesize
  Repeat this procedure of selecting and adding edges until\\
  the right graph has $N-1$ edges, where $N$ is the number of clusters.
}

\begin{figure}

\begin{subfigure}[b]{0.49\linewidth}
\centering
\scalebox{0.8}{
\begin{tikzpicture}[>=latex]
  
  % The various elements are conveniently placed using a matrix:
  \matrix[row sep=1.0cm,column sep=1.0cm,ampersand replacement=\&] {

    % First line
    \node (bd)  [bag, minimum size=1.8cm, draw=D] {$\{B_4,D_7\}$};    \&
    \node (bef) [bag, minimum size=1.8cm, draw=F] {$\{B_4,E_2,F_3\}$}; \\
    % Second line
    \node (bce) [bag, minimum size=1.8cm, draw=E] {$\{B_4,C_3,E_2\}$};  \&
    \node (abc) [bag, minimum size=1.8cm, draw=C] {$\{A_3,B_4,C_3\}$}; \\
  };

  % The diagram elements are now connected through lines:
  % (See manual pg. 132)
  \path[-]
    (bd)  edge node[above] {$1$} (bef)
          edge node[left] {$1$} (bce)
          edge node[above right,pos=0.8] {$1$} (abc)
    %(bef) edge node[above left,pos=0.8] {$2$} (bce)
    (bef) edge node[right] {$1$} (abc)
    (bce) edge node[below] {$2$} (abc)
    ;

\end{tikzpicture}
}
\end{subfigure}
\hfill
\begin{subfigure}[b]{0.49\linewidth}
\centering
\scalebox{0.8}{
\begin{tikzpicture}[>=latex]
  
  % The various elements are conveniently placed using a matrix:
  \matrix[row sep=1.0cm,column sep=1.0cm,ampersand replacement=\&] {

    % First line
    \node (bd)  [bag, minimum size=1.8cm, draw=D] {$\{B_4,D_7\}$};    \&
    \node (bef) [bag, minimum size=1.8cm, draw=F] {$\{B_4,E_2,F_3\}$}; \\
    % Second line
    \node (bce) [bag, minimum size=1.8cm, draw=E] {$\{B_4,C_3,E_2\}$};  \&
    \node (abc) [bag, minimum size=1.8cm, draw=C] {$\{A_3,B_4,C_3\}$}; \\
  };
  \path[-]
    (bce) edge (bef)
    ;
\end{tikzpicture}
}
\end{subfigure}

\end{figure}

\end{frame}

%------------------------------------------------------------------------------
%%
%------------------------------------------------------------------------------

\begin{frame}[fragile]{Connection of Clusters}

\centering

%{\footnotesize
%  Repeat selecting and adding edges until the right graph has $N-1$ edges.
%}

\begin{figure}

\begin{subfigure}[b]{0.49\linewidth}
\centering
\scalebox{0.8}{
\begin{tikzpicture}[>=latex]
  
  % The various elements are conveniently placed using a matrix:
  \matrix[row sep=1.0cm,column sep=1.0cm,ampersand replacement=\&] {

    % First line
    \node (bd)  [bag, minimum size=1.8cm, draw=D] {$\{B_4,D_7\}$};    \&
    \node (bef) [bag, minimum size=1.8cm, draw=F] {$\{B_4,E_2,F_3\}$}; \\
    % Second line
    \node (bce) [bag, minimum size=1.8cm, draw=E] {$\{B_4,C_3,E_2\}$};  \&
    \node (abc) [bag, minimum size=1.8cm, draw=C] {$\{A_3,B_4,C_3\}$}; \\
  };

  % The diagram elements are now connected through lines:
  % (See manual pg. 132)
  \path[-]
    (bd)  edge node[above] {$1$} (bef)
          edge node[left] {$1$} (bce)
          edge node[above right,pos=0.8] {$1$} (abc)
    %(bef) edge [ultra thick] node[above left,pos=0.8] {$2$} (bce)
    (bef) edge node[right] {$1$} (abc)
    (bce) edge [ultra thick] node[below] {$2$} (abc)
    ;

\end{tikzpicture}
}
\end{subfigure}
\hfill
\begin{subfigure}[b]{0.49\linewidth}
\centering
\scalebox{0.8}{
\begin{tikzpicture}[>=latex]
  
  % The various elements are conveniently placed using a matrix:
  \matrix[row sep=1.0cm,column sep=1.0cm,ampersand replacement=\&] {

    % First line
    \node (bd)  [bag, minimum size=1.8cm, draw=D] {$\{B_4,D_7\}$};    \&
    \node (bef) [bag, minimum size=1.8cm, draw=F] {$\{B_4,E_2,F_3\}$}; \\
    % Second line
    \node (bce) [bag, minimum size=1.8cm, draw=E] {$\{B_4,C_3,E_2\}$};  \&
    \node (abc) [bag, minimum size=1.8cm, draw=C] {$\{A_3,B_4,C_3\}$}; \\
  };
  \path[-]
    (bce) edge (bef)
    ;
\end{tikzpicture}
}
\end{subfigure}

\end{figure}

\end{frame}

%------------------------------------------------------------------------------
%%
%------------------------------------------------------------------------------

\begin{frame}[fragile]{Connection of Clusters}

\centering

%{\footnotesize
%  Repeat selecting and adding edges until the right graph has $N-1$ edges.
%}

\begin{figure}

\begin{subfigure}[b]{0.49\linewidth}
\centering
\scalebox{0.8}{
\begin{tikzpicture}[>=latex]
  
  % The various elements are conveniently placed using a matrix:
  \matrix[row sep=1.0cm,column sep=1.0cm,ampersand replacement=\&] {

    % First line
    \node (bd)  [bag, minimum size=1.8cm, draw=D] {$\{B_4,D_7\}$};    \&
    \node (bef) [bag, minimum size=1.8cm, draw=F] {$\{B_4,E_2,F_3\}$}; \\
    % Second line
    \node (bce) [bag, minimum size=1.8cm, draw=E] {$\{B_4,C_3,E_2\}$};  \&
    \node (abc) [bag, minimum size=1.8cm, draw=C] {$\{A_3,B_4,C_3\}$}; \\
  };

  % The diagram elements are now connected through lines:
  % (See manual pg. 132)
  \path[-]
    (bd)  edge node[above] {$1$} (bef)
          edge node[left] {$1$} (bce)
          edge node[above right,pos=0.8] {$1$} (abc)
    %(bef) edge [ultra thick] node[above left,pos=0.8] {$2$} (bce)
    (bef) edge node[right] {$1$} (abc)
    %(bce) edge [ultra thick] node[below] {$2$} (abc)
    ;

\end{tikzpicture}
}
\end{subfigure}
\hfill
\begin{subfigure}[b]{0.49\linewidth}
\centering
\scalebox{0.8}{
\begin{tikzpicture}[>=latex]
  
  % The various elements are conveniently placed using a matrix:
  \matrix[row sep=1.0cm,column sep=1.0cm,ampersand replacement=\&] {

    % First line
    \node (bd)  [bag, minimum size=1.8cm, draw=D] {$\{B_4,D_7\}$};    \&
    \node (bef) [bag, minimum size=1.8cm, draw=F] {$\{B_4,E_2,F_3\}$}; \\
    % Second line
    \node (bce) [bag, minimum size=1.8cm, draw=E] {$\{B_4,C_3,E_2\}$};  \&
    \node (abc) [bag, minimum size=1.8cm, draw=C] {$\{A_3,B_4,C_3\}$}; \\
  };
  \path[-]
    (bef) edge (bce)
    (bce) edge (abc)
    ;
\end{tikzpicture}
}
\end{subfigure}

\end{figure}

\end{frame}

%------------------------------------------------------------------------------
%%
%------------------------------------------------------------------------------

\begin{frame}[fragile]{Connection of Clusters}

\centering

%{\footnotesize
%  Repeat selecting and adding edges until the right graph has $N-1$ edges.
%}

\begin{figure}

\begin{subfigure}[b]{0.49\linewidth}
\centering
\scalebox{0.8}{
\begin{tikzpicture}[>=latex]
  
  % The various elements are conveniently placed using a matrix:
  \matrix[row sep=1.0cm,column sep=1.0cm,ampersand replacement=\&] {

    % First line
    \node (bd)  [bag, minimum size=1.8cm, draw=D] {$\{B_4,D_7\}$};    \&
    \node (bef) [bag, minimum size=1.8cm, draw=F] {$\{B_4,E_2,F_3\}$}; \\
    % Second line
    \node (bce) [bag, minimum size=1.8cm, draw=E] {$\{B_4,C_3,E_2\}$};  \&
    \node (abc) [bag, minimum size=1.8cm, draw=C] {$\{A_3,B_4,C_3\}$}; \\
  };

  % The diagram elements are now connected through lines:
  % (See manual pg. 132)
  \path[-]
    (bd)  edge [ultra thick] node[above] {$1$} (bef)
          edge [ultra thick] node[left] {$1$} (bce)
          edge [ultra thick] node[above right,pos=0.8] {$1$} (abc)
    %(bef) edge [ultra thick] node[above left,pos=0.8] {$2$} (bce)
    (bef) edge node[right] {$1$} (abc)
    %(bce) edge [ultra thick] node[below] {$2$} (abc)
    ;

\end{tikzpicture}
}
\end{subfigure}
\hfill
\begin{subfigure}[b]{0.49\linewidth}
\centering
\scalebox{0.8}{
\begin{tikzpicture}[>=latex]
  
  % The various elements are conveniently placed using a matrix:
  \matrix[row sep=1.0cm,column sep=1.0cm,ampersand replacement=\&] {

    % First line
    \node (bd)  [bag, minimum size=1.8cm, draw=D] {$\{B_4,D_7\}$};    \&
    \node (bef) [bag, minimum size=1.8cm, draw=F] {$\{B_4,E_2,F_3\}$}; \\
    % Second line
    \node (bce) [bag, minimum size=1.8cm, draw=E] {$\{B_4,C_3,E_2\}$};  \&
    \node (abc) [bag, minimum size=1.8cm, draw=C] {$\{A_3,B_4,C_3\}$}; \\
  };
  \path[-]
    (bef) edge (bce)
    (bce) edge (abc)
    ;
\end{tikzpicture}
}
\end{subfigure}

\end{figure}

\end{frame}

%------------------------------------------------------------------------------
%%
%------------------------------------------------------------------------------

\begin{frame}[fragile]{Connection of Clusters}

\centering

%{\footnotesize
%  Repeat selecting and adding edges until the right graph has $N-1$ edges.
%}

\begin{figure}

\begin{subfigure}[b]{0.49\linewidth}
\centering
\scalebox{0.8}{
\begin{tikzpicture}[>=latex]
  
  % The various elements are conveniently placed using a matrix:
  \matrix[row sep=1.0cm,column sep=1.0cm,ampersand replacement=\&] {

    % First line
    \node (bd)  [bag, minimum size=1.8cm, draw=D] {$\{B_4,D_7\}$};    \&
    \node (bef) [bag, minimum size=1.8cm, draw=F] {$\{B_4,E_2,F_3\}$}; \\
    % Second line
    \node (bce) [bag, minimum size=1.8cm, draw=E] {$\{B_4,C_3,E_2\}$};  \&
    \node (abc) [bag, minimum size=1.8cm, draw=C] {$\{A_3,B_4,C_3\}$}; \\
  };

  % The diagram elements are now connected through lines:
  % (See manual pg. 132)
  \path[-]
    (bd)  edge node[above] {$1$} (bef)
          edge [ultra thick] node[left] {$1$} (bce)
          edge node[above right,pos=0.8] {$1$} (abc)
    %(bef) edge [ultra thick] node[above left,pos=0.8] {$2$} (bce)
    (bef) edge node[right] {$1$} (abc)
    %(bce) edge [ultra thick] node[below] {$2$} (abc)
    ;

\end{tikzpicture}
}
\end{subfigure}
\hfill
\begin{subfigure}[b]{0.49\linewidth}
\centering
\scalebox{0.8}{
\begin{tikzpicture}[>=latex]
  
  % The various elements are conveniently placed using a matrix:
  \matrix[row sep=1.0cm,column sep=1.0cm,ampersand replacement=\&] {

    % First line
    \node (bd)  [bag, minimum size=1.8cm, draw=D] {$\{B_4,D_7\}$};    \&
    \node (bef) [bag, minimum size=1.8cm, draw=F] {$\{B_4,E_2,F_3\}$}; \\
    % Second line
    \node (bce) [bag, minimum size=1.8cm, draw=E] {$\{B_4,C_3,E_2\}$};  \&
    \node (abc) [bag, minimum size=1.8cm, draw=C] {$\{A_3,B_4,C_3\}$}; \\
  };
  \path[-]
    (bef) edge (bce)
    (bce) edge (abc)
    ;
\end{tikzpicture}
}
\end{subfigure}

\end{figure}

\end{frame}

%------------------------------------------------------------------------------
%%
%------------------------------------------------------------------------------

\begin{frame}[fragile]{Connection of Clusters}

\centering

%{\footnotesize
%  Repeat selecting and adding edges until the right graph has $N-1$ edges.
%}

\begin{figure}

\begin{subfigure}[b]{0.49\linewidth}
\centering
\scalebox{0.8}{
\begin{tikzpicture}[>=latex]
  
  % The various elements are conveniently placed using a matrix:
  \matrix[row sep=1.0cm,column sep=1.0cm,ampersand replacement=\&] {

    % First line
    \node (bd)  [bag, minimum size=1.8cm, draw=D] {$\{B_4,D_7\}$};    \&
    \node (bef) [bag, minimum size=1.8cm, draw=F] {$\{B_4,E_2,F_3\}$}; \\
    % Second line
    \node (bce) [bag, minimum size=1.8cm, draw=E] {$\{B_4,C_3,E_2\}$};  \&
    \node (abc) [bag, minimum size=1.8cm, draw=C] {$\{A_3,B_4,C_3\}$}; \\
  };

  % The diagram elements are now connected through lines:
  % (See manual pg. 132)
  \path[-]
    (bd)  edge node[above] {$1$} (bef)
          %edge [ultra thick] node[left] {$1$} (bce)
    (bd)  edge node[above right,pos=0.8] {$1$} (abc)
    %(bef) edge [ultra thick] node[above left,pos=0.8] {$2$} (bce)
    (bef) edge node[right] {$1$} (abc)
    %(bce) edge [ultra thick] node[below] {$2$} (abc)
    ;

\end{tikzpicture}
}
\end{subfigure}
\hfill
\begin{subfigure}[b]{0.49\linewidth}
\centering
\scalebox{0.8}{
\begin{tikzpicture}[>=latex]
  
  % The various elements are conveniently placed using a matrix:
  \matrix[row sep=1.0cm,column sep=1.0cm,ampersand replacement=\&] {

    % First line
    \node (bd)  [bag, minimum size=1.8cm, draw=D] {$\{B_4,D_7\}$};    \&
    \node (bef) [bag, minimum size=1.8cm, draw=F] {$\{B_4,E_2,F_3\}$}; \\
    % Second line
    \node (bce) [bag, minimum size=1.8cm, draw=E] {$\{B_4,C_3,E_2\}$};  \&
    \node (abc) [bag, minimum size=1.8cm, draw=C] {$\{A_3,B_4,C_3\}$}; \\
  };
  \path[-]
    (bef) edge (bce)
    (bce) edge (abc)
    (bd) edge (bce)
    ;
\end{tikzpicture}
}
\end{subfigure}

\end{figure}

\end{frame}

%------------------------------------------------------------------------------
%%
%------------------------------------------------------------------------------

\begin{frame}[fragile]{Connection of Clusters}

\centering

{\footnotesize
  Finally, we label each edge with a \textit{sepset},
  \\i.e. the intersection of variables between adjacent clusters.
}

\begin{figure}

\begin{subfigure}[b]{0.41\linewidth}
\centering
\scalebox{0.75}{
\begin{tikzpicture}[>=latex]
  
  % The various elements are conveniently placed using a matrix:
  \matrix[row sep=1.0cm,column sep=1.0cm,ampersand replacement=\&] {

    % First line
    \node (bd)  [bag, minimum size=1.8cm, draw=D] {$\{B_4,D_7\}$};    \&
    \node (bef) [bag, minimum size=1.8cm, draw=F] {$\{B_4,E_2,F_3\}$}; \\
    % Second line
    \node (bce) [bag, minimum size=1.8cm, draw=E] {$\{B_4,C_3,E_2\}$};  \&
    \node (abc) [bag, minimum size=1.8cm, draw=C] {$\{A_3,B_4,C_3\}$}; \\
  };
  \path[-]
    (bef) edge (bce)
    (bce) edge (abc)
    (bd) edge (bce)
    ;
\end{tikzpicture}
}
\end{subfigure}
\hfill
\begin{subfigure}[b]{0.58\linewidth}
\centering
\scalebox{0.75}{
\begin{tikzpicture}[>=latex]

  \tikzset{CliqueStyle/.style =  {opacity=.4,color=#1,}}
  
  % The various elements are conveniently placed using a matrix:
  \matrix[row sep=0.26cm,column sep=0.20cm,ampersand replacement=\&] {
    % First line
                                                         \&
                                                         \&
    \node (bd) [bag, minimum size=1.8cm, draw=D] {$\{B,D\}$};  \&
                                                         \&
                                                        \\
    % Second line
                                                         \&
                                                         \&
    \node (b) [hsepset] {$\{B\}$};                    \&
                                                         \&
                                                        \\
    % Third line
    \node (abc) [bag,minimum size=1.8cm,draw=C] {$\{A,B,C\}$};  \&
    \node (bc) [vsepset] {$\{B,C\}$};                    \&
    \node (bce) [bag,minimum size=1.8cm,draw=E] {$\{B,C,E\}$};  \&
    \node (be) [vsepset] {$\{B,E\}$};                    \&
    \node (bef) [bag,minimum size=1.8cm,draw=F] {$\{B,E,F\}$};
                                                        \\
  };

  % The diagram elements are now connected through lines:
  \path[-]
    (bd) edge (b)
    (b) edge (bce)
    (abc) edge (bc)
    (bc) edge (bce)
    (bce) edge (be)
    (be) edge (bef)
    ;

  %% Draw colored circles inside clusters corresponding to the assigned cliques
  %\node at (bd) [draw,yshift=-4.0mm] [bag,CliqueStyle=MaterialGreen] {} ;
  %\node at (abc) [draw,yshift= 5.0mm] [bag,CliqueStyle=MaterialOrange] {} ;
  %\node at (abc) [draw,yshift=-4.0mm,xshift= 3.0mm] [bag,CliqueStyle=MaterialBlue] {} ;
  %\node at (abc) [draw,yshift=-4.0mm,xshift=-3.0mm] [bag,CliqueStyle=MaterialCyan] {} ;
  %\node at (bce) [draw,yshift=-4.0mm] [bag,CliqueStyle=MaterialPurple] {} ;
  %\node at (bef) [draw,yshift=-4.0mm] [bag,CliqueStyle=MaterialRed] {} ;

\end{tikzpicture}
}
\end{subfigure}

\end{figure}

\end{frame}

